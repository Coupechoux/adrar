\documentclass{article}
\usepackage[T1]{fontenc}
\usepackage[utf8]{inputenc}
\usepackage[french]{babel}
\usepackage{listings}
\usepackage{color}
\definecolor{lightgray}{rgb}{.9,.9,.9}
\definecolor{darkgray}{rgb}{.4,.4,.4}
\definecolor{purple}{rgb}{0.65, 0.12, 0.82}

\lstdefinelanguage{JavaScript}{
  keywords={typeof, new, true, false, catch, function, return, null, catch, switch, var, if, in, while, do, else, case, break},
  keywordstyle=\color{blue}\bfseries,
  ndkeywords={class, export, boolean, throw, implements, import, this},
  ndkeywordstyle=\color{darkgray}\bfseries,
  identifierstyle=\color{black},
  sensitive=false,
  comment=[l]{//},
  morecomment=[s]{/*}{*/},
  commentstyle=\color{purple}\ttfamily,
  stringstyle=\color{red}\ttfamily,
  morestring=[b]',
  morestring=[b]"
}

\lstset{
   language=JavaScript,
   backgroundcolor=\color{lightgray},
   extendedchars=true,
   basicstyle=\footnotesize\ttfamily,
   showstringspaces=false,
   showspaces=false,
   numbers=left,
   numberstyle=\footnotesize,
   numbersep=9pt,
   tabsize=2,
   breaklines=true,
   showtabs=false,
   captionpos=b
}

\pagestyle{empty}

\begin{document}
	\section*{Groupe 1}
	\begin{lstlisting}[escapeinside={(*}{*)}]
		function groupe_1(t) {
			let s = "";
			for(let i = 0 ; i < t.length ; i++) {
				let c = t[i];
				let n = 4;
				if(c[0] == "J") {
					n = 1;
				} else if(c[0] == "Q") {
					n = 2;
				} else if(c[0] == "K") {
					n = 3;
				}
				n *= 5;
				if(c[1] == "(*$\heartsuit$*)") {
					n += 1;
				} else if(c[1] == "(*$\diamondsuit$*)") {
					n += 2;
				} else if(c[1] == "(*$\clubsuit$*)") {
					n += 3;
				} else {
					n += 4;
				}
				s += groupe_2(n);
			}
			return s;
		}
	\end{lstlisting}
	Entrée du programme : t - un tableau de chaînes de caractères. 
	
	\vspace{1em}
	\noindent Un tableau est une collection de plusieurs variables. t.length correspond à la taille du tableau, c'est-à-dire au nombre de variables à l'intérieur.
	t[0] correspond à la première variable, t[1] à la suivante, etc.
	
	Exemple :
	\begin{lstlisting}
		t = ["Bonjour", 42, 3.5];
	\end{lstlisting}
	$t.length$ vaut $3$.\\$t[0]$ vaut $"Bonjour"$.\\$t[1]$ vaut $42$.\\$t[2]$ vaut $3.5$.
	\newpage
	\section*{Groupe 2}
	\begin{lstlisting}[escapeinside={(*}{*)}]
		function groupe_2(n) {
			let s = "";
			for(let i = 0 ; i < 6 ; i++) {
				s += n%2;
				n = Math.floor(n/2); // Division entiere, on oublie les chiffres apres la virgule
			}
			s = groupe_3(s);
			return s;
		}
	\end{lstlisting}
	Entrée du programme : un entier positif $n$.
	
	\vspace{1em}
	\noindent Rappel : $n\%2 \longrightarrow$ n modulo 2. C'est le reste de la division euclidienne (=~entière) de $n$ par $2$. C'est combien il me reste une fois que j'ai enlevé $2$ autant que je pouvais à $n$.
	
	\newpage
	\section*{Groupe 3}
	\begin{lstlisting}[escapeinside={(*}{*)}]
		function groupe_3(s) {
			let result = "";
			for(let i = 0 ; i < s.length ; i++) {
				result += s[s.length-1-i];
			}
			return result;
		}
	\end{lstlisting}
	Entrée du programme : une chaîne de caractères $s$.
	
	\vspace{1em}
	\noindent $s.length$ correspond à la taille de $s$, c'est-à-dire à son nombre de lettres/symboles. $s[0]$ correspond à la première lettre de $s$, $s[1]$ à la deuxième, etc.
	
	\newpage
	\section*{Groupe 4}
	\begin{lstlisting}[escapeinside={(*}{*)}]
		function groupe_4(s,p) {
			let result = "";
			for(let i = 6*p ; i < 6*p+6 ; i++) {
				result += s[i];
			}
			return result;
		}
	\end{lstlisting}
	Entrées du programme :
	\begin{enumerate}
		\item Une chaîne de caractères $s$ ;
		\item Un entier positif $p$.
	\end{enumerate}
	
	\vspace{1em}
	\noindent $s[0]$ correspond à la première lettre de $s$, $s[1]$ à la deuxième, etc.
		
	\newpage
	\section*{Groupe 5}
	\begin{lstlisting}[escapeinside={(*}{*)}]
		function groupe_5(s) {
			let n = 0;
			for(let i = 0 ; i < s.length ; i++) {
				n = 2*n + parseInt(s[i]);
			}
			return n;
		}
	\end{lstlisting}
	Entrée du programme : une chaîne de caractères $s$.
	
	\vspace{1em}
	\noindent $s.length$ correspond à la taille de $s$, c'est-à-dire à son nombre de lettres/symboles. $s[0]$ correspond à la première lettre de $s$, $s[1]$ à la deuxième, etc.\\
	$parseInt$ permet de convertir une chaîne de caractères en un entier. Par exemple, $parseInt("1")$ vaut $1$.
	
	
	\newpage
	\section*{Groupe 6}
	\begin{lstlisting}[escapeinside={(*}{*)}]
		function groupe_6(s) {
			for(let p = 0 ; p < s.length/6 ; p++) {
				let b = groupe_4(s,p);
				let n = groupe_5(b);
				let n2 = n%5;
				let n1 = (n/5)%5;
				if(n1 == 1) {
					ecrire_au_tableau("Valet de ");
				} else if(n1 == 2) {
					ecrire_au_tableau("Dame de ");
				} else if(n1 == 3) {
					ecrire_au_tableau("Roi de ");
				} else {
					ecrire_au_tableau("As de ");
				}
		
				if(n2 == 1) {
					ecrire_au_tableau("coeur");
				} else if(n2 == 2) {
					ecrire_au_tableau("carreau");
				} else if(n2 == 3) {
					ecrire_au_tableau("trefle");
				} else {
					ecrire_au_tableau("pique");
				}
				ecrire_au_tableau("\n"); // \n correspond a un retour a la ligne
			}
		}
	\end{lstlisting}
	Entrée du programme : une chaîne de caractères $s$.
	
	\vspace{1em}
	\noindent $s.length$ correspond à la taille de $s$, c'est-à-dire à son nombre de lettres/symboles.
	
	\newpage
	\section*{Groupe 7}
	\begin{lstlisting}[escapeinside={(*}{*)}]
		function groupe_7(t) {
			let b = groupe_1(t);	
			ecrire_au_tableau("Code = " + b);
			groupe_6(b);
		}
	\end{lstlisting}
	Entrée du programme : un tableau de chaînes de caractères $t$.
	
	\vspace{1em}
	\noindent Un tableau est une collection de plusieurs variables. t.length correspond à la taille du tableau, c'est-à-dire au nombre de variables à l'intérieur.

\end{document}