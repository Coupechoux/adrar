\documentclass{article}
\usepackage{xcolor}
\usepackage{listings}
\usepackage{algorithm,algpseudocode}
\usepackage[utf8]{inputenc}

\lstdefinelanguage{JavaScript}{
  keywords={typeof, new, true, false, catch, function, return, null, catch, switch, var, if, in, while, do, else, case, break},
  keywordstyle=\color{blue}\bfseries,
  ndkeywords={class, export, boolean, throw, implements, import, this},
  ndkeywordstyle=\color{darkgray}\bfseries,
  identifierstyle=\color{black},
  sensitive=false,
  comment=[l]{//},
  morecomment=[s]{/*}{*/},
  commentstyle=\color{purple}\ttfamily,
  stringstyle=\color{red}\ttfamily,
  morestring=[b]',
  morestring=[b]"
}

\lstset{
   language=JavaScript,
   backgroundcolor=\color{lightgray},
   extendedchars=true,
   basicstyle=\footnotesize\ttfamily,
   showstringspaces=false,
   showspaces=false,
   numbers=left,
   numberstyle=\footnotesize,
   numbersep=9pt,
   tabsize=2,
   breaklines=true,
   showtabs=false,
   captionpos=b
}

\title{Boucle \emph{for}}
\date{}

\begin{document}
	\fboxsep0pt
	\maketitle
	\section{Syntaxe}	
	Dans plusieurs langages, la syntaxe de la boucle \emph{for} est la suivante :
	
	\begin{center}
		\begin{minipage}{0.3\textwidth}
			$for(A;B;C)$ \{\\
			\hspace*{20px} $D;$\\
			\hspace*{20px} $E;$\\
			\}
		\end{minipage}
	\end{center}
	
	\section{Signification}
	\noindent$A$ : Initialisation\\
	$B$ : Condition\\
	$C$ : Fin de boucle\\
	$D$ : Corps de la boucle\\
	$E$ : Corps de la boucle\\
	
	\noindent Lorsque le programme "arrive" sur la boucle $for$, il suit le parcours suivant~:
	
	\begin{enumerate}
		\item  il commence par faire le $A$, l'initialisation.
		\item Ensuite, il teste la condition $B$.
		\item Si la condition est vraie :
		\begin{enumerate}
			\item il passe au corps de la boucle ; c'est ce qu'il y a entre les accolades ($D$ et $E$).
			\item Il fait la fin de la boucle : le $C$.
			\item Il retourne en 2, à la condition $B$.
		\end{enumerate}
		\item Si la condition est fausse, la boucle est terminée. Le programme \emph{ne} fait \emph{pas} ce qu'il y a dans les accolades, et passe directement à la suite.
	\end{enumerate}
	\section{Exemple (en js)}			
		\begin{lstlisting}
			let a = 3;
			for(let i = 0 ; i<7 ; i += 2) {
				console.log("Bonjour numero " + i);
				a *= 2;
			}
			console.log("Fin ! La variable a vaut " + a);
		\end{lstlisting}
		
		Le programme se déroule de la manière suivante :
		
		\begin{tabular}{|c|c|c|c|c|}
			\hline
			Ligne & Variable $a$ & Variable $i$ & Condition & Affichage écran\\
			\hline
			\hline
			1&3&&&\\
			\hline
			2 - initialisation&3&0&&\\
			\hline
			2 - condition&3&0&Vrai&\\
			\hline
			3&3&0&&Bonjour numero 0\\
			\hline
			4&6&0&&\\
			\hline
			2 - fin de boucle&6&2&&\\
			\hline
			2 - condition&6&2&Vrai&\\
			\hline
			3&6&2&&Bonjour numero 2\\
			\hline
			4&12&2&&\\
			\hline
			2 - fin de boucle&12&4&&\\
			\hline
			2 - condition&12&4&Vrai&\\
			\hline
			3&12&4&&Bonjour numero 4\\
			\hline
			4&24&4&&\\
			\hline
			2 - fin de boucle&24&6&&\\
			\hline
			2 - condition&24&6&Vrai&\\
			\hline
			3&24&6&&Bonjour numero 6\\
			\hline
			4&48&6&&\\
			\hline
			2 - fin de boucle&48&8&&\\
			\hline
			2 - condition&48&8&Faux&\\
			\hline
			6&48&&&Fin ! La variable a vaut 48\\
			\hline
		\end{tabular}
\end{document}